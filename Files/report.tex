\documentclass[12pt,fleqn,a4paper]{article}

\usepackage{latexsym}
\usepackage{url}
\usepackage{xspace}
\usepackage{epsfig}
\usepackage{psfrag}
\usepackage{a4wide}
\usepackage{marvosym}
\usepackage{amsmath,amsfonts,amssymb,amsthm,latexsym}
\usepackage{graphics,graphicx,color,subfigure}
\usepackage{fancyhdr}
\usepackage[english]{babel}
\usepackage[latin1]{inputenc}
% added this package solely because "our" title is way too long and it
% allows to add a line break in the title
%\usepackage[usestackEOL]{stackengine}

\textheight 680pt
\textwidth 460pt
\topmargin -40pt
\oddsidemargin 5pt
\evensidemargin 5pt
\parindent 0pt

\pagestyle{fancyplain} \setlength{\headheight}{16pt}
\renewcommand{\sectionmark}[1]{\markright{\thesection\ #1}}
\lhead[\fancyplain{}{\thepage}]
    {\fancyplain{}{\rightmark}}
\rhead[\fancyplain{}{\leftmark}]
    {\fancyplain{}{\thepage}}
\cfoot{}
\renewcommand{\thesection}{\arabic{section}}
\renewcommand{\thesubsection}{\arabic{section}.\arabic{subsection}}


\begin{document}
\begin{titlepage}%Institution
\vspace{2cm}
\centerline{
\large{Department of Computer Sciences}}
\vspace{0.2cm}
\centerline{\large{University of Salzburg}}%Title with one or two Lines(More if wanted)
%\hline
\vspace{2cm}

\centerline{\large{PS Natural Computation}}
\centerline{SS 13/14}
\vspace{1cm}

\centerline{\Large{Design and implementation of a robot task}}
%\centerline{\Large{\bf\Longstack{SIMMA\\Design and implementation of a robot task\\demonstration the effect of
%neuromodulators}} }%Type of the Document
\vspace{1cm}

\vspace{0.4cm}%Date
\centerline{\today}
\vspace{5cm}%Authors

%\hline
\vspace{0.2cm}
Project Members:\\
\centerline{Auinger, 1220321, auingerto@stud.sbg.ac.at}\\
\centerline{M\"{u}ller, 1123410, mueller110@gmx.net}\\
\centerline{Pollhammer, 9520061, pollhammerand@stud.sbg.ac.at}\\
\centerline{Schwarz, 1220024, schwarzst@stud.sbg.ac.at}\\
\vspace {0.8cm}\\

Academic Supervisor: \\
\centerline{Helmut MAYER}
\centerline{helmut@cosy.sbg.ac.at}
\vspace{0.8cm}\\
Correspondence to: \\
\centerline{Universit\"{a}t Salzburg} \\
\centerline{Fachbereich Computerwissenschaften} \\
\centerline{Jakob--Haringer--Stra\ss e 2} \\
\centerline{A--5020 Salzburg} \\
\centerline{Austria}
\clearpage
\end{titlepage}



%%%
%Table of Content
% \setcounter{page}{1}
% \pagenumbering{Roman} %I,II,III... in the TOC
% \tableofcontents

\clearpage
\pagestyle{headings}
\pagenumbering{arabic}  %Better if TOC is variable (more than one page)
\setcounter{page}{1}
\pagenumbering{arabic}  %Better if TOC is variable (more than one page)
\setcounter{page}{1}

\abstract
{The main goal of this project is designing a task that demonstrates that the usage of neuromodulators can influence the evolution of a neural network in a positive way. For implementation and simulation of the task we use SIMMA "a simulation framework mainly developed for the simulation of mobile autonomous robots and their behaviour".}

%%%
\section{Introduction}
In this class our team of four people design a task for neural networks.  The main goal is to demonstrate that the usage of neuromodulators can influence the evolution of a neural network in a positive way. After careful considerations we decided to implement a robot whose task it is to take several pegs to a certain place. By pushing the pegs to the certain place the robot has to avoid some enemies. If the robot is captured by an enemy, he will get penalty seconds. The robot will learn this task by means of neural networks.  We hope that neuromodulators will influence the evolution of our neural network in a positive way. So the neuromodulators assist our robot by learning this task. For implementation and simulation of the task we use SIMMA "a simulation framework mainly developed for the simulation of mobile autonomous robots and their behaviour".
\newpage


% links go here, NOT in references

%%%
\section{Links}

\begin{itemize}
	\item Project Page: \url{http://student.cosy.sbg.ac.at/~cmueller/natcomp/}
	\item PS Page:	\url{http://www.cosy.sbg.ac.at/~helmut/Teaching/NaturalComputation/proseminar.html}
\end{itemize}

\bibliography{}		% .bib files here



%
% end of document
%
\end{document}
